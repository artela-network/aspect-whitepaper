\section{Aspect-Boost}
Aspect-Boost is an initial implementation of Aspect Programming on Artela, and it is a framework for any kind of blockchain to integrate to enable Aspect Programming on it. With Aspect-Boost, blockchain can support users to develop, deploy and execute Aspect on it. To integrate Aspect-Boost, blockchain nodes need to implement the adaptors of Aspect-Boots.

\subsection{Overview}

Aspect-Boots contains the following key components to boost the process of Aspect integration with different blockchains:

\begin{figure}[h]
  \centering
  \includegraphics[width=0.4\textwidth]{sections/aspect-boost.png}
  \caption{Aspect-Boost Key Components}
\end{figure}

\begin{itemize}
  \item \textbf{Aspect Core.} The Aspect core is the management module within the Aspect framework and is primarily implemented through smart contracts. It mainly comprises protocol contracts and precompiled contracts. The Aspect core manages major lifecycle processes of Aspects, including deployment, upgrade, and binding, among others. This core module meticulously scrutinizes the ownership of the Aspect in relation to the operation initiator's address, maintains the binding relationships between Aspects and smart contracts, and handles the token settlement for Aspects.
  
  \item \textbf{JPM Controller.} The join point Model controller acts as a hub, connecting the blockchain base layer and the various components of the Aspect framework. It comes into action when specific transaction or block lifecycle stages are reached. Upon the activation of a join point, the JPM controller retrieves the relevant Aspects from the Aspect core, using the address of the smart contract that invoked the action. It then constructs an instance of the Aspect runtime (ArtWASM) to execute the Aspect. In addition, the JPM controller functions as a translator, mediating interactions between the base layer and the Aspect. When an Aspect is invoking base layer functions, the JPM controller's adaptor translates this system call into the corresponding request compatible with the blockchain platform, ensuring the successful execution of the call.
  
  \item \textbf{ArtWASM.} ArtWASM is a tailored WebAssembly (WASM) runtime specifically built for executing Aspects. Functioning as a sandbox, ArtWASM ensures the secure and efficient execution of Aspects within the blockchain environment. Its design incorporates significant optimizations, aligning it perfectly with the unique requirements of Aspect execution. ArtWASM serves as a bridge between Aspects, system modules, and the JPM controller by providing a comprehensive suite of runtime APIs. These APIs act as the conduit, enabling the necessary communication between these entities. When an Aspect initiates a system call, ArtWASM routes this call to the appropriate module, thereby facilitating seamless integration and interaction across the different components of the Aspect framework.
  
  \item \textbf{System Modules.} System modules comprise a set of system-level modules that can be seamlessly integrated with the blockchain base layer. These modules enhance the functionalities of dApps, enabling them to utilize the full potential of Aspects.
  
  \item \textbf{RPC Server.} The RPC server provides a set of interfaces that are compatible with existing blockchain platforms. These interfaces facilitate the processing of Aspect operations, ensuring interoperability between different platforms.
  
  \item \textbf{Aspect SDK.} The Aspect SDK equips developers with a set of tools to build rich dApps using Aspects. It simplifies the process of incorporating Aspects into dApps and allows for more advanced and flexible applications.
\end{itemize}

\subsection{Aspect Core}

The Aspect core operates as a precompiled system contract, crafted in native code, and acts as the critical module governing Aspect life cycles. We've chosen to implement the Aspect core as a smart contract for these primary reasons:

\begin{itemize}
  \item Simpler interaction with smart contracts and the underlying world state: The Aspect core's nature as a smart contract simplifies processes such as binding authorization checks. For instance, invoking the validation interface of the smart contract is a mandatory step, which is made more straightforward by treating Aspect core as a smart contract.
  \item Compatibility with existing transaction data structure: Implementing Aspect core as a smart contract eliminates the need to create new types of transactions specifically for Aspect management. This ensures the interfaces are completely compatible with the current ones, providing seamless integration.
\end{itemize}

\textbf{Aspect Management.} The Aspect core oversees changes to the state that influence the state transition function of an Aspect. When executing such operations like deploy, upgrade, or destroy, It meticulously checks the ownership of the Aspect, validating the address of the operation initiator against that of the Aspect's governor. Only the governor of the Aspect is permitted to modify the behavior of the Aspect's state transition functions.

The Aspect core meticulously manages alterations to the state that affect an Aspect's state transition function. It performs a careful assessment of the Aspect's ownership, cross-verifying the address of the operation initiator with the address of the Aspect's governor. It's critical to understand that only the governor of the Aspect has the authority to make modifications to the behavior of the Aspect's state transition functions.

\textbf{Binding Relationship Management.} The binding relationships between Aspects and smart contracts are stored within the state of the Aspect core. Any changes to these binding relationships are reflected in the state of the Aspect core. When the blockchain receives a transaction, the Aspect core identifies the Aspects associated with the recipient's address (provided the recipient is a smart contract) and initiates the execution of the relevant join points.

\textbf{Fee Settlement.} Aspects, not being typical blockchain account objects, require a settlement account to handle token transactions. The Aspect core manages this process, as Aspects can't directly modify the balance of the settlement account. The settlement account sets an allowance amount of tokens for a given Aspect's fee settlements, and the Aspect core ensures that fees are deducted only if there is sufficient allowance.

\subsection{ArtWASM}

ArtWASM is a secure and deterministic WASM Runtime implementation and it is designed for executing Aspect. It adheres to the design of the WASM Virtual Machine - a stack-based virtual machine constructed to execute WASM binary code.

\textbf{Gas metering.} Similar to a smart contract, gas metering has been implemented in ArtWASM to quantify resource usage by Aspect. This section will outline the implementation of gas metering in the WASM environment, drawing parallels with the gas metering models used in EVM and other blockchain execution environments.

We implemented gas metering in WASM via byte code instrumentation. This involves inserting additional instructions into the WASM code to monitor and control gas consumption.

Each WASM operation can be assigned a gas cost reflecting its complexity or the resources it uses. During the instrumentation process, additional WASM instructions are inserted before each operation to deduct the operation's gas cost from the total gas available. If the operation's gas cost exceeds the remaining gas, execution is halted, preventing excessive resource usage.

Consider a simple WASM function that adds two integers. In raw WASM, this might be represented by a sequence of instructions that load the integers from memory, perform the addition, and store the result back in memory.

To instrument this function with gas metering, additional instructions would be inserted before each operation. These instructions would deduct the gas cost of the operation from the total gas available. For example, if loading an integer from memory has a gas cost of 1, and performing an addition has a gas cost of 2, the instrumented WASM code would deduct these amounts from the total gas before performing the operations.

Instrumenting gas rules in WASM can bring a new level of resource management to the platform, drawing on principles similar to those used in Move and other blockchain platforms. While the specifics would need to be tailored to the WASM environment and the specific needs of its users, the underlying concept of gas as a means of metering resource usage is a powerful one that could greatly enhance the capabilities of the WASM platform.

\textbf{Gas rule for Aspect.} The gas model of eWASM is our main reference for ArtWASM. Endowed with an extensive selection of 64-bit Integer operations, data type alterations, and operations that are type-parametric, each operation incurs a unique gas expense to accurately mirror the computational exertion involved.

Every WASM opcode has a corresponding Intel IA-32 (x86) opcode (or a series of opcodes) assigned to it. These opcodes have a fixed cycle count, which Intel refers to as latency. This methodology uses a particular CPU model from the Sky Lake architecture (Intel Xeon Platinum 8175M, one of the most used CPUs by AWS EC2), similar to CPUs fabricated in 2018, to provide a reasonable average depiction of the Artela nodes. It is presumed that a 2.5 Ghz model embodies the average, with the clock rate approximately equaling 2,500,000,000 cycles per second. An additional supposition establishes that 1 second of CPU execution is equivalent to 10 million gas, meaning that 1 gas equals 0.1 us. This leads to 0.004 gas per cycle. Moreover, it is stipulated that the gas costs undergo regular adjustments, ideally every three years, in tandem with the consistent enhancements in CPUs and the routine upgrade of hardware for Artela nodes.

The WASM opcodes demand less computational power in contrast to EVM opcodes. EVM opcodes work on 256 bits of data, while WASM opcodes are restricted to a maximum of 64 bits. This means it takes four instructions, at the very least, to match EVM. Most arithmetic instructions in EVM cost 3 gas, translating to 0.75 gas for most 64-bit WASM instructions. This variance in processing requirements brings forth the notion of particles. Within the system, Aspect gas measurements are logged in a 64-bit variable with 4 decimal digits precision. When transitioning the WASM gas count to EVM gas, it is divided by 10,000 and rounded upwards. If the outcome is less than 0, it is then adjusted to 1.

\textbf{Runtime Pool.} During the lifecycle of a transaction or smart contract call, an instance of Aspect will be invoked multiple times. If the runtime instance is reinstantiated every time, it will bring significant delays to the execution. The following is some data we have measured:

\begin{center}
\begin{tabular}{|c|c|}
  \hline
  Method & Time Cost (micro-seconds) \\
  \hline
  NewEngine & 276 \\
  NewStore & 43 \\
  NewModule & 6395 \\
  NewLinker & 26 \\
  Function Link & 1034 \\
  \hline
\end{tabular}
\end{center}

ArtWASM's built-in runtime pool has made these overheads negligible by caching the runtime instances in an LRU cache and reusing the instances when needed.

\subsection{JPM Controller}

The JPM Controller is a pivotal component in the Aspect Boost, serving as the central hub between the blockchain base layer and Aspects. One of its main roles is providing a join point adaptor, an interface designed for integration with various blockchain platforms. This adaptor offers a set of interfaces akin to the blockchain base layer, allowing Aspects to be triggered at the appropriate stages throughout the block and transaction lifecycle.

Additionally, blockchains are required to implement a set of interfaces defined by the join point adaptor. This allows the join point controller to make system calls to the appropriate base layer modules, ensuring seamless interaction between the two layers.

The join point Controller also plays a crucial role in Aspect runtime construction, taking responsibility for the assembly of the Aspect runtime instance. It injects the appropriate runtime context and host API functions into the Aspect runtime instance based on the specifications of different join points. This guarantees the correct execution of functions within the Aspect, thereby ensuring the system's successful operation.

\subsection{System Modules}

The Scheduler is a module designed to automate transaction submissions to the current block proposal. This functionality simplifies the production of on-chain automation, like automated price feeds, thereby lessening protocol dependence on off-chain networks and enhancing dApp decentralization. The Aspect system can schedule transactions in either the current or future blocks, enabling full automation of protocols with on-chain procedures. However, it's important to note that the execution of scheduled transactions is not guaranteed. In cases where the assigned block space is full, transactions may need to be rescheduled.

\textbf{ArtEVM}

ArtEVM is an enhanced version of EVM provided in Aspect-Boost. ArtEVM is able to track all the state variable changes when a smart contract is executing, which provides an overview of account states for Aspect. In the meantime, ArtEVM can execute smart contract calls in parallel, maximizing the execution performance.

An improved version of the SOLC compiler has been implemented to actualize this feature. Additional state tracing opcode will be incorporated into the EVM bytecode using the pattern outlined below:

\begin{verbatim}
...
JOURNAL <-- Records the state before the SSTORE operation
SSTORE  <-- EVM Opcode that executes a change to the world state
JOURNAL <-- Records the state after the SSTORE operation
...
\end{verbatim}

By employing this pattern, the new instructions can systematically document state changes both preceding and following the SSTORE operation. With the help of the \texttt{journal} opcodes, the recorded state changes will save in the EVM context with the following format:

ArtEVM is capable of executing EVM smart contract calls in parallel. Traditionally, the execution of transactions in EVM is sequential due to potential data conflicts. However, ArtEVM leverages techniques including multi-version state-trie and block STM algorithm to analyze potential dependencies and conflicts between transactions, enabling them to be processed simultaneously without conflict. By grouping non-conflicting transactions together and processing them in parallel, ArtEVM can significantly increase the throughput and efficiency of smart contract execution. This parallel execution capability of ArtEVM enables it to handle a greater volume of transactions, thus boosting the scalability of the blockchain network.

ArtEVM is fully compatible with Ethereum's EVM. The state tracing feature can be toggled on or off by the compiler, providing developers with a choice: they can opt for a more cost-efficient smart contract or a more secure one.

\subsection{Aspect SDK}

The Aspect SDK is a comprehensive toolkit that empowers developers to build sophisticated, dynamic decentralized applications (dApps) using Aspects. The SDK currently provides the following tools to streamline the integration of Aspects into dApps.

\begin{itemize}
  \item \textbf{RPC Client:} This tool allows developers to manage and interact with Aspects through code snippets. Since native blockchain platforms do not directly support Aspect operations, the RPC Client provides a set of interfaces related to Aspect management, mirroring those provided by the RPC Server.
  
  \item \textbf{Aspect Tool:} This command-line tool manages the entire lifecycle of Aspect development. It assists developers from creating an Aspect project to publishing, covering every stage of the Aspect development lifecycle.
  
  \item \textbf{Aspect Lib:} This collection of WASM libraries provides the basic Aspects. It includes a variety of components such as cryptographic libraries (e.g., hashing and signature verification), utility libraries (e.g., format conversions and encoding/decoding tools), and host API libraries (enabling system calls to system modules within the Aspect framework), among others.
  
  \item \textbf{ArtSOLC:} An enhanced Solidity compiler (SOLC) version, ArtSOLC can instrument state and call stack tracing code into EVM bytecode. This makes it possible to monitor state and call stacks in Aspects, further expanding the capabilities of Aspects in the context of dApps.
\end{itemize}

\subsection{RPC Server}

The RPC Server is an essential component in the Aspect framework, providing a set of interfaces compatible with existing blockchain platforms for processing Aspect-related operations. The RPC Server is responsible for handling all Aspect-related requests, such as deploying an Aspect, binding an Aspect to a smart contract, unbinding, upgrading, and others. It translates these operations into a compatible format that can be processed by the underlying blockchain base layer.
